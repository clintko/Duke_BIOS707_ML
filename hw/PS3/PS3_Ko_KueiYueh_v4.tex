\documentclass[]{article}
\usepackage{lmodern}
\usepackage{amssymb,amsmath}
\usepackage{ifxetex,ifluatex}
\usepackage{fixltx2e} % provides \textsubscript
\ifnum 0\ifxetex 1\fi\ifluatex 1\fi=0 % if pdftex
  \usepackage[T1]{fontenc}
  \usepackage[utf8]{inputenc}
\else % if luatex or xelatex
  \ifxetex
    \usepackage{mathspec}
  \else
    \usepackage{fontspec}
  \fi
  \defaultfontfeatures{Ligatures=TeX,Scale=MatchLowercase}
\fi
% use upquote if available, for straight quotes in verbatim environments
\IfFileExists{upquote.sty}{\usepackage{upquote}}{}
% use microtype if available
\IfFileExists{microtype.sty}{%
\usepackage{microtype}
\UseMicrotypeSet[protrusion]{basicmath} % disable protrusion for tt fonts
}{}
\usepackage[margin=1in]{geometry}
\usepackage{hyperref}
\hypersetup{unicode=true,
            pdftitle={BIOS707 \textbar{} Problem set3},
            pdfauthor={Kuei-Yueh (Clint) Ko},
            pdfborder={0 0 0},
            breaklinks=true}
\urlstyle{same}  % don't use monospace font for urls
\usepackage{color}
\usepackage{fancyvrb}
\newcommand{\VerbBar}{|}
\newcommand{\VERB}{\Verb[commandchars=\\\{\}]}
\DefineVerbatimEnvironment{Highlighting}{Verbatim}{commandchars=\\\{\}}
% Add ',fontsize=\small' for more characters per line
\usepackage{framed}
\definecolor{shadecolor}{RGB}{248,248,248}
\newenvironment{Shaded}{\begin{snugshade}}{\end{snugshade}}
\newcommand{\KeywordTok}[1]{\textcolor[rgb]{0.13,0.29,0.53}{\textbf{#1}}}
\newcommand{\DataTypeTok}[1]{\textcolor[rgb]{0.13,0.29,0.53}{#1}}
\newcommand{\DecValTok}[1]{\textcolor[rgb]{0.00,0.00,0.81}{#1}}
\newcommand{\BaseNTok}[1]{\textcolor[rgb]{0.00,0.00,0.81}{#1}}
\newcommand{\FloatTok}[1]{\textcolor[rgb]{0.00,0.00,0.81}{#1}}
\newcommand{\ConstantTok}[1]{\textcolor[rgb]{0.00,0.00,0.00}{#1}}
\newcommand{\CharTok}[1]{\textcolor[rgb]{0.31,0.60,0.02}{#1}}
\newcommand{\SpecialCharTok}[1]{\textcolor[rgb]{0.00,0.00,0.00}{#1}}
\newcommand{\StringTok}[1]{\textcolor[rgb]{0.31,0.60,0.02}{#1}}
\newcommand{\VerbatimStringTok}[1]{\textcolor[rgb]{0.31,0.60,0.02}{#1}}
\newcommand{\SpecialStringTok}[1]{\textcolor[rgb]{0.31,0.60,0.02}{#1}}
\newcommand{\ImportTok}[1]{#1}
\newcommand{\CommentTok}[1]{\textcolor[rgb]{0.56,0.35,0.01}{\textit{#1}}}
\newcommand{\DocumentationTok}[1]{\textcolor[rgb]{0.56,0.35,0.01}{\textbf{\textit{#1}}}}
\newcommand{\AnnotationTok}[1]{\textcolor[rgb]{0.56,0.35,0.01}{\textbf{\textit{#1}}}}
\newcommand{\CommentVarTok}[1]{\textcolor[rgb]{0.56,0.35,0.01}{\textbf{\textit{#1}}}}
\newcommand{\OtherTok}[1]{\textcolor[rgb]{0.56,0.35,0.01}{#1}}
\newcommand{\FunctionTok}[1]{\textcolor[rgb]{0.00,0.00,0.00}{#1}}
\newcommand{\VariableTok}[1]{\textcolor[rgb]{0.00,0.00,0.00}{#1}}
\newcommand{\ControlFlowTok}[1]{\textcolor[rgb]{0.13,0.29,0.53}{\textbf{#1}}}
\newcommand{\OperatorTok}[1]{\textcolor[rgb]{0.81,0.36,0.00}{\textbf{#1}}}
\newcommand{\BuiltInTok}[1]{#1}
\newcommand{\ExtensionTok}[1]{#1}
\newcommand{\PreprocessorTok}[1]{\textcolor[rgb]{0.56,0.35,0.01}{\textit{#1}}}
\newcommand{\AttributeTok}[1]{\textcolor[rgb]{0.77,0.63,0.00}{#1}}
\newcommand{\RegionMarkerTok}[1]{#1}
\newcommand{\InformationTok}[1]{\textcolor[rgb]{0.56,0.35,0.01}{\textbf{\textit{#1}}}}
\newcommand{\WarningTok}[1]{\textcolor[rgb]{0.56,0.35,0.01}{\textbf{\textit{#1}}}}
\newcommand{\AlertTok}[1]{\textcolor[rgb]{0.94,0.16,0.16}{#1}}
\newcommand{\ErrorTok}[1]{\textcolor[rgb]{0.64,0.00,0.00}{\textbf{#1}}}
\newcommand{\NormalTok}[1]{#1}
\usepackage{graphicx,grffile}
\makeatletter
\def\maxwidth{\ifdim\Gin@nat@width>\linewidth\linewidth\else\Gin@nat@width\fi}
\def\maxheight{\ifdim\Gin@nat@height>\textheight\textheight\else\Gin@nat@height\fi}
\makeatother
% Scale images if necessary, so that they will not overflow the page
% margins by default, and it is still possible to overwrite the defaults
% using explicit options in \includegraphics[width, height, ...]{}
\setkeys{Gin}{width=\maxwidth,height=\maxheight,keepaspectratio}
\IfFileExists{parskip.sty}{%
\usepackage{parskip}
}{% else
\setlength{\parindent}{0pt}
\setlength{\parskip}{6pt plus 2pt minus 1pt}
}
\setlength{\emergencystretch}{3em}  % prevent overfull lines
\providecommand{\tightlist}{%
  \setlength{\itemsep}{0pt}\setlength{\parskip}{0pt}}
\setcounter{secnumdepth}{0}
% Redefines (sub)paragraphs to behave more like sections
\ifx\paragraph\undefined\else
\let\oldparagraph\paragraph
\renewcommand{\paragraph}[1]{\oldparagraph{#1}\mbox{}}
\fi
\ifx\subparagraph\undefined\else
\let\oldsubparagraph\subparagraph
\renewcommand{\subparagraph}[1]{\oldsubparagraph{#1}\mbox{}}
\fi

%%% Use protect on footnotes to avoid problems with footnotes in titles
\let\rmarkdownfootnote\footnote%
\def\footnote{\protect\rmarkdownfootnote}

%%% Change title format to be more compact
\usepackage{titling}

% Create subtitle command for use in maketitle
\newcommand{\subtitle}[1]{
  \posttitle{
    \begin{center}\large#1\end{center}
    }
}

\setlength{\droptitle}{-2em}

  \title{BIOS707 \textbar{} Problem set3}
    \pretitle{\vspace{\droptitle}\centering\huge}
  \posttitle{\par}
    \author{Kuei-Yueh (Clint) Ko}
    \preauthor{\centering\large\emph}
  \postauthor{\par}
    \date{}
    \predate{}\postdate{}
  

\begin{document}
\maketitle

{
\setcounter{tocdepth}{2}
\tableofcontents
}
\section{About the notebook}\label{about-the-notebook}

This the problem set 3 of the course BIOS707. Below are the libraries
required for this report.

\begin{Shaded}
\begin{Highlighting}[]
\NormalTok{### tools}
\KeywordTok{library}\NormalTok{(tidyverse)}
\end{Highlighting}
\end{Shaded}

\begin{verbatim}
## -- Attaching packages ----------------------------------------------------------------------------------------------------- tidyverse 1.2.1 --
\end{verbatim}

\begin{verbatim}
## √ ggplot2 3.0.0     √ purrr   0.2.5
## √ tibble  1.4.2     √ dplyr   0.7.6
## √ tidyr   0.8.1     √ stringr 1.3.1
## √ readr   1.1.1     √ forcats 0.3.0
\end{verbatim}

\begin{verbatim}
## -- Conflicts -------------------------------------------------------------------------------------------------------- tidyverse_conflicts() --
## x dplyr::filter() masks stats::filter()
## x dplyr::lag()    masks stats::lag()
\end{verbatim}

\begin{Shaded}
\begin{Highlighting}[]
\KeywordTok{library}\NormalTok{(cluster)}
\KeywordTok{library}\NormalTok{(foreach)}
\end{Highlighting}
\end{Shaded}

\begin{verbatim}
## 
## Attaching package: 'foreach'
\end{verbatim}

\begin{verbatim}
## The following objects are masked from 'package:purrr':
## 
##     accumulate, when
\end{verbatim}

\begin{Shaded}
\begin{Highlighting}[]
\KeywordTok{library}\NormalTok{(clue)}
\KeywordTok{library}\NormalTok{(pROC)}
\end{Highlighting}
\end{Shaded}

\begin{verbatim}
## Type 'citation("pROC")' for a citation.
\end{verbatim}

\begin{verbatim}
## 
## Attaching package: 'pROC'
\end{verbatim}

\begin{verbatim}
## The following objects are masked from 'package:stats':
## 
##     cov, smooth, var
\end{verbatim}

\begin{Shaded}
\begin{Highlighting}[]
\NormalTok{### plot and display}
\KeywordTok{library}\NormalTok{(RColorBrewer)}
\KeywordTok{library}\NormalTok{(pheatmap)}
\KeywordTok{library}\NormalTok{(grid)}
\KeywordTok{library}\NormalTok{(gridExtra)}
\end{Highlighting}
\end{Shaded}

\begin{verbatim}
## 
## Attaching package: 'gridExtra'
\end{verbatim}

\begin{verbatim}
## The following object is masked from 'package:dplyr':
## 
##     combine
\end{verbatim}

\section{Part I. Simulation}\label{part-i.-simulation}

\subsection{Q1. Explore Clustering}\label{q1.-explore-clustering}

\textbf{Simulate a dataset where n = 100 and p = 50. Used your ICC
function from problem set 2 to create 3 (k) distinct clusters in
dataset.}

\begin{Shaded}
\begin{Highlighting}[]
\NormalTok{##### Global variables #####}
\NormalTok{K          <-}\StringTok{ }\DecValTok{3}    \CommentTok{# number of group}
\NormalTok{N          <-}\StringTok{ }\DecValTok{100}  \CommentTok{# number of observations}
\NormalTok{SIG2_ALPHA <-}\StringTok{ }\DecValTok{10}   \CommentTok{# variance of alpha for icc simulation}
\NormalTok{P          <-}\StringTok{ }\DecValTok{50}   \CommentTok{# number of parameters}
\end{Highlighting}
\end{Shaded}

\begin{Shaded}
\begin{Highlighting}[]
\NormalTok{##### My ICC function #####}
\NormalTok{simulate_icc <-}\StringTok{ }\ControlFlowTok{function}\NormalTok{(icc, }\DataTypeTok{grp_size =} \DecValTok{10}\NormalTok{, }\DataTypeTok{n_grp =} \DecValTok{5}\NormalTok{, }\DataTypeTok{mu_tot =} \DecValTok{0}\NormalTok{, }\DataTypeTok{sig2_a =} \DecValTok{1}\NormalTok{)\{}
    \CommentTok{# function to simuate the data from specific ICC}
    \CommentTok{# =================================}
    
\NormalTok{    ### calculate sig2_e from the ICC and sig2_a}
\NormalTok{    sig2_e <-}\StringTok{ }\NormalTok{sig2_a }\OperatorTok{*}\StringTok{ }\NormalTok{(}\DecValTok{1} \OperatorTok{/}\StringTok{ }\NormalTok{icc }\OperatorTok{-}\StringTok{ }\DecValTok{1}\NormalTok{)}
    
\NormalTok{    ### simulate alpha by sig2_a}
\NormalTok{    alpha  <-}\StringTok{ }\KeywordTok{rnorm}\NormalTok{(n_grp, }\DataTypeTok{mean =} \DecValTok{0}\NormalTok{, }\DataTypeTok{sd =}\NormalTok{ sig2_a}\OperatorTok{^}\FloatTok{0.5}\NormalTok{)}
    
\NormalTok{    ### simulate error for each group by sig2_e}
\NormalTok{    df <-}\StringTok{ }\KeywordTok{sapply}\NormalTok{(mu_tot }\OperatorTok{+}\StringTok{ }\NormalTok{alpha, }\ControlFlowTok{function}\NormalTok{(x)\{}
\NormalTok{        eps <-}\StringTok{ }\KeywordTok{rnorm}\NormalTok{(grp_size, }\DataTypeTok{mean =} \DecValTok{0}\NormalTok{, }\DataTypeTok{sd =}\NormalTok{ sig2_e}\OperatorTok{^}\FloatTok{0.5}\NormalTok{)}
        \KeywordTok{return}\NormalTok{(x }\OperatorTok{+}\StringTok{ }\NormalTok{eps)}
\NormalTok{    \})}

\NormalTok{    ### set column names}
\NormalTok{    df <-}\StringTok{ }\KeywordTok{data.frame}\NormalTok{(df)}
    \KeywordTok{colnames}\NormalTok{(df) <-}\StringTok{ }\KeywordTok{paste}\NormalTok{(}\StringTok{"Grp"}\NormalTok{, }\DecValTok{1}\OperatorTok{:}\NormalTok{n_grp)}

\NormalTok{    ### gather and return}
\NormalTok{    df <-}\StringTok{ }\NormalTok{df }\OperatorTok\StringTok{ }\KeywordTok{gather}\NormalTok{(Group, Value)}
    \KeywordTok{return}\NormalTok{(df)}
\NormalTok{\} }\CommentTok{# end func}
\end{Highlighting}
\end{Shaded}

\subsubsection{Q1 (a)}\label{q1-a}

\textbf{Set your ICC to 0.3 and simulate a dataset. Show your simulation
code. Apply K-means clustering and evaluate how well your clusters
uncover the true group}

First, I set required function for data simulation and visualization

\begin{Shaded}
\begin{Highlighting}[]
\NormalTok{##### helper function in simulate data #####}
\NormalTok{my_combine <-}\StringTok{ }\ControlFlowTok{function}\NormalTok{(x, y)\{}
    \CommentTok{# binary merge function}
    \CommentTok{# ==========================}
\NormalTok{    y =}\StringTok{ }\NormalTok{y }\OperatorTok\StringTok{ }\KeywordTok{select}\NormalTok{(}\OperatorTok{-}\NormalTok{Group)}
    \KeywordTok{return}\NormalTok{(}\KeywordTok{bind_cols}\NormalTok{(x, y))}
\NormalTok{\} }\CommentTok{# end func}

\NormalTok{##### function that simulate the data #####}
\NormalTok{simulate_data <-}\StringTok{ }\ControlFlowTok{function}\NormalTok{(icc, }\DataTypeTok{k =}\NormalTok{ K, }\DataTypeTok{n =}\NormalTok{ N, }\DataTypeTok{sig2_a =}\NormalTok{ SIG2_ALPHA, }\DataTypeTok{p =}\NormalTok{ P, }\DataTypeTok{var_name =} \StringTok{"V"}\NormalTok{, }\DataTypeTok{func_combine =}\NormalTok{ my_combine)\{}
    \CommentTok{# simulate data using the icc simulating function above}
    \CommentTok{# =======================================}
    
\NormalTok{    ### simulation using icc function}
\NormalTok{    dat =}\StringTok{ }\KeywordTok{replicate}\NormalTok{(}
\NormalTok{        p,             }\CommentTok{# number of parameters}
        \KeywordTok{simulate_icc}\NormalTok{( }\CommentTok{# each parameter is simulate independently}
\NormalTok{            icc, }
            \DataTypeTok{n_grp =}\NormalTok{ k, }
            \DataTypeTok{grp_size =} \KeywordTok{floor}\NormalTok{(n }\OperatorTok{/}\StringTok{ }\NormalTok{k), }
            \DataTypeTok{sig2_a =}\NormalTok{ sig2_a),}
        \DataTypeTok{simplify =} \OtherTok{FALSE}\NormalTok{)}
    
\NormalTok{    ### reduced simulated data into one dataframe}
\NormalTok{    dat =}\StringTok{ }\KeywordTok{foreach}\NormalTok{(}
         \DataTypeTok{idx =} \DecValTok{1}\OperatorTok{:}\KeywordTok{length}\NormalTok{(dat), }
         \DataTypeTok{.combine =}\NormalTok{ func_combine) }\OperatorTok\StringTok{ }\NormalTok{\{}
\NormalTok{             df =}\StringTok{ }\NormalTok{dat[[idx]]}
\NormalTok{             df =}\StringTok{ }\NormalTok{df }\OperatorTok\StringTok{ }\NormalTok{dplyr}\OperatorTok{::}\KeywordTok{rename_}\NormalTok{(}
                 \DataTypeTok{.dots=}\KeywordTok{setNames}\NormalTok{(}
                \KeywordTok{names}\NormalTok{(.), }
                \KeywordTok{c}\NormalTok{(}\StringTok{"Group"}\NormalTok{, }\KeywordTok{paste0}\NormalTok{(var_name, idx))))}
\NormalTok{    \} }\CommentTok{# end do in foreach}
    
    \KeywordTok{return}\NormalTok{(dat)}
\NormalTok{\} }\CommentTok{# end func}

\NormalTok{##### Function to visualize the results #####}
\NormalTok{plot_dat_sim <-}\StringTok{ }\ControlFlowTok{function}\NormalTok{(dat, variable)\{}
    \CommentTok{# visualize simulated data with specify variables}
    \CommentTok{# ========================================}
    
\NormalTok{    ### arrange the wide data frame into long format}
\NormalTok{    df <-}\StringTok{ }\NormalTok{dat }\OperatorTok\StringTok{ }
\StringTok{        }\KeywordTok{select}\NormalTok{(Group, variable) }\OperatorTok\StringTok{ }
\StringTok{        }\KeywordTok{gather}\NormalTok{(variable, value, }\OperatorTok{-}\NormalTok{Group)}
    
\NormalTok{    ### ggplot each variables color by group}
\NormalTok{    gp <-}\StringTok{ }\KeywordTok{ggplot}\NormalTok{(df, }\KeywordTok{aes}\NormalTok{(}\DataTypeTok{x =}\NormalTok{ Group, }\DataTypeTok{y =}\NormalTok{ value, }\DataTypeTok{color =}\NormalTok{ Group)) }\OperatorTok{+}\StringTok{ }
\StringTok{        }\KeywordTok{geom_jitter}\NormalTok{(}\DataTypeTok{width =} \FloatTok{0.2}\NormalTok{) }\OperatorTok{+}
\StringTok{        }\KeywordTok{theme}\NormalTok{(}
            \DataTypeTok{axis.title.y =} \KeywordTok{element_text}\NormalTok{(}\DataTypeTok{size =} \DecValTok{10}\NormalTok{),}
            \DataTypeTok{axis.title.x =} \KeywordTok{element_text}\NormalTok{(}\DataTypeTok{size =} \DecValTok{10}\NormalTok{),}
            \DataTypeTok{axis.text.x  =} \KeywordTok{element_text}\NormalTok{(}\DataTypeTok{size =} \DecValTok{12}\NormalTok{, }\DataTypeTok{angle =} \DecValTok{90}\NormalTok{, }\DataTypeTok{vjust =} \FloatTok{0.5}\NormalTok{),}
            \DataTypeTok{strip.text.x =} \KeywordTok{element_text}\NormalTok{(}\DataTypeTok{size =} \DecValTok{10}\NormalTok{, }\DataTypeTok{color =} \StringTok{"grey30"}\NormalTok{)) }\OperatorTok{+}
\StringTok{        }\KeywordTok{facet_wrap}\NormalTok{(}\OperatorTok{~}\NormalTok{variable)}
    
    \KeywordTok{return}\NormalTok{(gp)}
\NormalTok{\} }\CommentTok{# end func}
\end{Highlighting}
\end{Shaded}

Using the function defined above, I am able to simulate data using any
specify ICC and number of parameters. Below is just an example to
demonstrate the functions work properly
\includegraphics{PS3_Ko_KueiYueh_v4_files/figure-latex/unnamed-chunk-5-1.pdf}

Simulate data with ICC 0.3

\begin{Shaded}
\begin{Highlighting}[]
\NormalTok{### initialization}
\KeywordTok{set.seed}\NormalTok{(}\DecValTok{0}\NormalTok{)}

\NormalTok{### simulation}
\NormalTok{dat_sim =}\StringTok{ }\KeywordTok{simulate_data}\NormalTok{(}\FloatTok{0.5}\NormalTok{)}

\NormalTok{### visualization}
\NormalTok{gp =}\StringTok{ }\KeywordTok{plot_dat_sim}\NormalTok{(dat_sim, }\KeywordTok{c}\NormalTok{(}\StringTok{"V1"}\NormalTok{, }\StringTok{"V10"}\NormalTok{, }\StringTok{"V50"}\NormalTok{)) }\OperatorTok{+}
\StringTok{    }\KeywordTok{ggtitle}\NormalTok{(}\StringTok{"Simulating random effects with ICC 0.3"}\NormalTok{)}
\KeywordTok{print}\NormalTok{(gp)}
\end{Highlighting}
\end{Shaded}

\includegraphics{PS3_Ko_KueiYueh_v4_files/figure-latex/unnamed-chunk-6-1.pdf}

To perform kmean clustering and evaluate how well the clusters uncover
the true groups, I implement a function that accept data and number of
times to perform kmeans clustering. For each kmeans clustering, a
confusion matrix between true group label and cluster id from kmeans is
calculated. For the confusion matrix, a Hungarian algorithm is applied
to perform cluster alignment. For a perfect clustering case, all the
points in one cluster belongs to the same group. This calulation than
allows me to get the points that are not cluster correctly, which I
define them as \textbf{unmatch}.

\begin{Shaded}
\begin{Highlighting}[]
\NormalTok{##### function to simulate kmeans clustering with several times #####}
\NormalTok{simulate_kmean <-}\StringTok{ }\ControlFlowTok{function}\NormalTok{(dat, iteration, }\DataTypeTok{is_mat_conf =} \OtherTok{FALSE}\NormalTok{)\{}
\NormalTok{    ### initialization}
\NormalTok{    unmatch =}\StringTok{ }\KeywordTok{c}\NormalTok{()}
\NormalTok{    mat =}\StringTok{ }\NormalTok{dat }\OperatorTok\StringTok{ }\KeywordTok{select}\NormalTok{(}\OperatorTok{-}\NormalTok{Group) }\OperatorTok\StringTok{ }\NormalTok{as.matrix}
    
\NormalTok{    ### perform multiple times of kmean clustering}
    \ControlFlowTok{for}\NormalTok{ (idx }\ControlFlowTok{in} \DecValTok{1}\OperatorTok{:}\NormalTok{iteration)\{}
\NormalTok{        ### get cluster id from kmeans}
\NormalTok{        clust =}\StringTok{ }\KeywordTok{kmeans}\NormalTok{(mat, }\DataTypeTok{centers =} \DecValTok{3}\NormalTok{)}
    
\NormalTok{        ### confusion matrix}
\NormalTok{        mat_conf =}\StringTok{ }\KeywordTok{table}\NormalTok{(clust}\OperatorTok{$}\NormalTok{cluster, dat_sim}\OperatorTok{$}\NormalTok{Group)}
    
\NormalTok{        ### cluster alignment}
\NormalTok{        idx =}\StringTok{ }\KeywordTok{solve_LSAP}\NormalTok{(mat_conf, }\DataTypeTok{maximum =} \OtherTok{TRUE}\NormalTok{)}
        
\NormalTok{        ### calculate number of "unmatch" points}
\NormalTok{        num =}\StringTok{ }\KeywordTok{sum}\NormalTok{(mat_conf[}\KeywordTok{cbind}\NormalTok{(}\KeywordTok{seq_along}\NormalTok{(idx), idx)])}
\NormalTok{        unmatch =}\StringTok{ }\KeywordTok{c}\NormalTok{(unmatch, }\KeywordTok{nrow}\NormalTok{(mat) }\OperatorTok{-}\StringTok{ }\NormalTok{num) }
\NormalTok{    \} }\CommentTok{# end for loop}
    
    \ControlFlowTok{if}\NormalTok{ (is_mat_conf) \{}
\NormalTok{        ### just to demonstrate what does }
\NormalTok{        ### the confusion matrix looks like}
        \KeywordTok{return}\NormalTok{(}\KeywordTok{list}\NormalTok{(mat_conf, unmatch))}
        
\NormalTok{    \} }\ControlFlowTok{else}\NormalTok{ \{}
\NormalTok{        ### normal situation, return the unmatch results}
        \KeywordTok{return}\NormalTok{(unmatch)}
\NormalTok{    \} }\CommentTok{# end if-else}
    
\NormalTok{\} }\CommentTok{# end func}
\end{Highlighting}
\end{Shaded}

Evaluate one kmean clustering using the function defined above

\begin{Shaded}
\begin{Highlighting}[]
\NormalTok{### initialization}
\KeywordTok{set.seed}\NormalTok{(}\DecValTok{0}\NormalTok{)}

\NormalTok{### simulation}
\NormalTok{dat_sim =}\StringTok{ }\KeywordTok{simulate_data}\NormalTok{(}\FloatTok{0.5}\NormalTok{)}

\NormalTok{### kmeans and evaluation}
\NormalTok{res =}\StringTok{ }\KeywordTok{simulate_kmean}\NormalTok{(dat_sim, }\DecValTok{1}\NormalTok{, }\DataTypeTok{is_mat_conf =} \OtherTok{TRUE}\NormalTok{)}
\end{Highlighting}
\end{Shaded}

from the confution matrix, I know that the kmean I just ran assgin
cluster labels correctly compared to the true label

\begin{Shaded}
\begin{Highlighting}[]
\KeywordTok{print}\NormalTok{(res[[}\DecValTok{1}\NormalTok{]])}
\end{Highlighting}
\end{Shaded}

\begin{verbatim}
##    
##     Grp 1 Grp 2 Grp 3
##   1     0    33     0
##   2     0     0    33
##   3    33     0     0
\end{verbatim}

Since it is a ``perfect'' clustering, there is no ``unmatch'' points

\begin{Shaded}
\begin{Highlighting}[]
\KeywordTok{cat}\NormalTok{(res[[}\DecValTok{2}\NormalTok{]])}
\end{Highlighting}
\end{Shaded}

\begin{verbatim}
## 0
\end{verbatim}

Of course, we know that clustering is sometimes sensitive to the intial
centers. Therefore, here I define two indices calculated from the
variable ``unmatch'' by running multiple times of kmean clustering

\begin{itemize}
\tightlist
\item
  \textbf{expected unmatch event}: if a clustering results has more than
  10\% unmatch points (in our case, since there are 100 data points, the
  threshold is 10), this clustering fails to cluster points correctly
  and an ``unmatch event'' occur. The number of unmatch event is divided
  by the number of times the kmean clustering is perform (defined as
  argument ``iteration'' in the function simulate\_kmean)
\item
  \textbf{expected number of unmatch points}: when an unmatch event
  occur, there might be only few points that are misclustered;
  therefore, it is important to quantify how well the clustering works
  by counting the average points that have group label uncovered
  uncorrectly throughout the multiple kmeans. The average is defined as
  expected number of unmatch points. Note that the standard deviation of
  the unmatch points is also calculated to better visualize the
  variation of the number of unmatch points.
\end{itemize}

\begin{Shaded}
\begin{Highlighting}[]
\NormalTok{#### function to calculate unmatch event and expected number of unmatch points}
\NormalTok{summary_unmatch <-}\StringTok{ }\ControlFlowTok{function}\NormalTok{(unmatch, }\DataTypeTok{threshold =} \DecValTok{10}\NormalTok{, }\DataTypeTok{icc =} \OtherTok{NULL}\NormalTok{, }\DataTypeTok{q =} \OtherTok{NULL}\NormalTok{)\{}
    
\NormalTok{    ### calculate unmatch event and expected number of unmatch points }
\NormalTok{    unmatch_events    =}\StringTok{ }\KeywordTok{mean}\NormalTok{(unmatch }\OperatorTok{>}\StringTok{ }\NormalTok{threshold)}
\NormalTok{    unmatch_points_mu =}\StringTok{ }\KeywordTok{mean}\NormalTok{(unmatch)}
\NormalTok{    unmatch_points_sd =}\StringTok{ }\KeywordTok{sd}\NormalTok{(unmatch)}
    
\NormalTok{    ### return the results}
\NormalTok{    is_exist_icc =}\StringTok{ }\OperatorTok{!}\KeywordTok{is.null}\NormalTok{(icc)}
\NormalTok{    is_exist_q   =}\StringTok{ }\OperatorTok{!}\KeywordTok{is.null}\NormalTok{(q)}
    
    \ControlFlowTok{if}\NormalTok{ (is_exist_icc }\OperatorTok{&}\StringTok{ }\NormalTok{is_exist_q)\{}
\NormalTok{        res =}\StringTok{ }\KeywordTok{c}\NormalTok{(unmatch_events, unmatch_points_mu, unmatch_points_sd, icc, q)}
        \KeywordTok{names}\NormalTok{(res) =}\StringTok{ }\KeywordTok{c}\NormalTok{(}\StringTok{"events"}\NormalTok{, }\StringTok{"points_mu"}\NormalTok{, }\StringTok{"points_sd"}\NormalTok{, }\StringTok{"icc"}\NormalTok{, }\StringTok{"q"}\NormalTok{)}
        
\NormalTok{    \} }\ControlFlowTok{else} \ControlFlowTok{if}\NormalTok{ (is_exist_icc) \{}
\NormalTok{        res =}\StringTok{ }\KeywordTok{c}\NormalTok{(unmatch_events, unmatch_points_mu, unmatch_points_sd, icc)}
        \KeywordTok{names}\NormalTok{(res) =}\StringTok{ }\KeywordTok{c}\NormalTok{(}\StringTok{"events"}\NormalTok{, }\StringTok{"points_mu"}\NormalTok{, }\StringTok{"points_sd"}\NormalTok{, }\StringTok{"icc"}\NormalTok{)}
        
\NormalTok{    \} }\ControlFlowTok{else} \ControlFlowTok{if}\NormalTok{ (is_exist_q) \{}
\NormalTok{        res =}\StringTok{ }\KeywordTok{c}\NormalTok{(unmatch_events, unmatch_points_mu, unmatch_points_sd, q)}
        \KeywordTok{names}\NormalTok{(res) =}\StringTok{ }\KeywordTok{c}\NormalTok{(}\StringTok{"events"}\NormalTok{, }\StringTok{"points_mu"}\NormalTok{, }\StringTok{"points_sd"}\NormalTok{, }\StringTok{"q"}\NormalTok{)}
        
\NormalTok{    \} }\ControlFlowTok{else}\NormalTok{ \{}
\NormalTok{        res =}\StringTok{ }\KeywordTok{c}\NormalTok{(unmatch_events, unmatch_points_mu, unmatch_points_sd)}
        \KeywordTok{names}\NormalTok{(res) =}\StringTok{ }\KeywordTok{c}\NormalTok{(}\StringTok{"events"}\NormalTok{, }\StringTok{"points_mu"}\NormalTok{, }\StringTok{"points_sd"}\NormalTok{)}
\NormalTok{    \}}
    
    \KeywordTok{return}\NormalTok{(res)}
\NormalTok{\}}
\end{Highlighting}
\end{Shaded}

Here I use one simulated data to demonstrate the idea

\begin{Shaded}
\begin{Highlighting}[]
\NormalTok{### initialization}
\KeywordTok{set.seed}\NormalTok{(}\DecValTok{0}\NormalTok{)}

\NormalTok{### simulate data}
\NormalTok{dat_sim =}\StringTok{ }\KeywordTok{simulate_data}\NormalTok{(}\FloatTok{0.3}\NormalTok{)}

\NormalTok{### clustering}
\NormalTok{unmatch =}\StringTok{ }\KeywordTok{simulate_kmean}\NormalTok{(dat_sim, }\DecValTok{100}\NormalTok{)}

\NormalTok{### visualization}
\KeywordTok{data.frame}\NormalTok{(}\DataTypeTok{x =} \DecValTok{1}\OperatorTok{:}\KeywordTok{length}\NormalTok{(unmatch), }\DataTypeTok{y =}\NormalTok{ unmatch) }\OperatorTok
\StringTok{    }\KeywordTok{ggplot}\NormalTok{(., }\KeywordTok{aes}\NormalTok{(}\DataTypeTok{x =}\NormalTok{ x, }\DataTypeTok{y =}\NormalTok{ y)) }\OperatorTok{+}
\StringTok{    }\KeywordTok{geom_line}\NormalTok{() }\OperatorTok{+}
\StringTok{    }\KeywordTok{geom_point}\NormalTok{() }\OperatorTok{+}
\StringTok{    }\KeywordTok{labs}\NormalTok{(}\DataTypeTok{x =} \StringTok{"iteration"}\NormalTok{, }\DataTypeTok{y =} \StringTok{"number of points mismatch"}\NormalTok{)}
\end{Highlighting}
\end{Shaded}

\includegraphics{PS3_Ko_KueiYueh_v4_files/figure-latex/unnamed-chunk-12-1.pdf}

by the plot above, we can easiy calculate the unmatch events is 9 / 100
= 0.09

\begin{Shaded}
\begin{Highlighting}[]
\KeywordTok{summary_unmatch}\NormalTok{(unmatch)}
\end{Highlighting}
\end{Shaded}

\begin{verbatim}
##    events points_mu points_sd 
##   0.09000   4.18000  13.36842
\end{verbatim}

\subsubsection{Q1 (b)}\label{q1-b}

\textbf{Simulate additional datasets, varying the ICC. Apply K-means
clustering with k = 3. Evaluate how well your able to uncover the true
cluster labels. Visualize your results and comment on the ``required''
ICC for effective clustering.}

By defining unmatch events and unmatch points, I am able to evaluate the
efficiency of kmean clustering for each datasets generated with
different ICC

\begin{Shaded}
\begin{Highlighting}[]
\NormalTok{### initialization}
\NormalTok{iccs =}\StringTok{ }\KeywordTok{seq}\NormalTok{(}\FloatTok{0.1}\NormalTok{, }\FloatTok{1.0}\NormalTok{, }\DataTypeTok{length.out =} \DecValTok{10}\NormalTok{)}
\NormalTok{ITER_KM =}\StringTok{ }\DecValTok{500}
\KeywordTok{set.seed}\NormalTok{(}\DecValTok{0}\NormalTok{)}

\NormalTok{### simulation and evaluation}
\NormalTok{res_km_icc =}\StringTok{ }\OtherTok{NULL}
\ControlFlowTok{for}\NormalTok{ (icc }\ControlFlowTok{in}\NormalTok{ iccs)\{}
    \KeywordTok{cat}\NormalTok{(}\StringTok{"icc:"}\NormalTok{, icc, }\StringTok{"}\CharTok{\textbackslash{}n}\StringTok{"}\NormalTok{)}
\NormalTok{    dat_sim =}\StringTok{ }\KeywordTok{simulate_data}\NormalTok{(icc)}
\NormalTok{    unmatch =}\StringTok{ }\KeywordTok{simulate_kmean}\NormalTok{(dat_sim, ITER_KM)}
\NormalTok{    unmatch_sum =}\StringTok{ }\KeywordTok{summary_unmatch}\NormalTok{(unmatch, }\DataTypeTok{icc =}\NormalTok{ icc)}
    
\NormalTok{    res_km_icc =}\StringTok{ }\KeywordTok{rbind}\NormalTok{(res_km_icc, unmatch_sum)}
\NormalTok{\} }\CommentTok{# end for loop}
\end{Highlighting}
\end{Shaded}

\begin{verbatim}
## icc: 0.1 
## icc: 0.2 
## icc: 0.3 
## icc: 0.4 
## icc: 0.5 
## icc: 0.6 
## icc: 0.7 
## icc: 0.8 
## icc: 0.9 
## icc: 1
\end{verbatim}

By visualizing the results, we could observe how unmatch events and
points varied throughout different kmean clustering. When ICC is 0.1,
none of the clustering results is perfect and about
10\textasciitilde{}15 points are always clustered uncorrectly. As ICC
increases from 0.1 to 0.2, probability of unmatch events drop
dramatically and once ICC reaches to 1.0, all the clustering results are
perfect and therefore, all points have cluster ids and true group labels
``matched'' correctly.
\includegraphics{PS3_Ko_KueiYueh_v4_files/figure-latex/unnamed-chunk-15-1.pdf}

\subsubsection{Q1 (c)}\label{q1-c}

\textbf{Choose the minimum ICC from above where you think you can
uncover the clusters. Keep n and p the same but vary the number of
variables that have an ICC. i.e.~some q \textless{} p will have an ICC
where the rest will have an ICC of 0. These variables with ICC = 0 are
called noise." Assess the impact of the amount of noise variables on
ability to recapture clusters.}

First, I set the function to add ``noisy'' parameters in data.

\begin{Shaded}
\begin{Highlighting}[]
\NormalTok{simulate_data_noise <-}\StringTok{ }\ControlFlowTok{function}\NormalTok{(icc, q, }\DataTypeTok{k =}\NormalTok{ K, }\DataTypeTok{n =}\NormalTok{ N, }\DataTypeTok{sig2_a =}\NormalTok{ SIG2_ALPHA, }\DataTypeTok{p =}\NormalTok{ P, }\DataTypeTok{func_sim_data =}\NormalTok{ simulate_data)\{}
    
\NormalTok{    ### assert if q is larger than p}
    \ControlFlowTok{if}\NormalTok{ (q }\OperatorTok{>}\StringTok{ }\NormalTok{p)\{}
        \KeywordTok{cat}\NormalTok{(}\StringTok{"ERROR: argument q > p"}\NormalTok{)}
\NormalTok{    \}}
    
\NormalTok{    ### special case}
    \ControlFlowTok{if}\NormalTok{ (q }\OperatorTok{==}\StringTok{ }\DecValTok{0}\NormalTok{)\{}
\NormalTok{        dat =}\StringTok{ }\KeywordTok{simulate_data}\NormalTok{(}\FloatTok{0.00001}\NormalTok{)}
        
\NormalTok{    \} }\ControlFlowTok{else} \ControlFlowTok{if}\NormalTok{ (q }\OperatorTok{==}\StringTok{ }\DecValTok{50}\NormalTok{) \{}
\NormalTok{        dat =}\StringTok{ }\KeywordTok{simulate_data}\NormalTok{(icc)}
        
\NormalTok{    \} }\ControlFlowTok{else}\NormalTok{ \{}
\NormalTok{    ### normal case where q != p}
\NormalTok{        n1 =}\StringTok{ }\NormalTok{q}
\NormalTok{        n2 =}\StringTok{ }\NormalTok{p}\OperatorTok{-}\NormalTok{q}
    
\NormalTok{        ### simulate two data: one with specified icc and another one with extreme small icc}
\NormalTok{        dat1 =}\StringTok{ }\KeywordTok{func_sim_data}\NormalTok{(icc,    k, n, sig2_a, n1, }\DataTypeTok{var_name =} \StringTok{"V"}\NormalTok{)}
\NormalTok{        dat2 =}\StringTok{ }\KeywordTok{func_sim_data}\NormalTok{(}\FloatTok{0.00001}\NormalTok{, k, n, sig2_a, n2, }\DataTypeTok{var_name =} \StringTok{"N"}\NormalTok{)}
        
\NormalTok{        ### combine the results}
\NormalTok{        tmp =}\StringTok{ }\NormalTok{dat2 }\OperatorTok\StringTok{ }\KeywordTok{select}\NormalTok{(}\OperatorTok{-}\NormalTok{Group)}
\NormalTok{        dat =}\StringTok{ }\KeywordTok{bind_cols}\NormalTok{(dat1, tmp)}
\NormalTok{    \} }\CommentTok{# end if else}
    
    \KeywordTok{return}\NormalTok{(dat)}
\NormalTok{\} }\CommentTok{# end func}
\end{Highlighting}
\end{Shaded}

Check if the simulation works properly

\begin{Shaded}
\begin{Highlighting}[]
\NormalTok{### initialization}
\KeywordTok{set.seed}\NormalTok{(}\DecValTok{0}\NormalTok{)}

\NormalTok{### simulation}
\NormalTok{dat_sim =}\StringTok{ }\KeywordTok{simulate_data_noise}\NormalTok{(}\FloatTok{0.1}\NormalTok{, }\DecValTok{10}\NormalTok{)}

\NormalTok{### visualization}
\NormalTok{gp =}\StringTok{ }\KeywordTok{plot_dat_sim}\NormalTok{(dat_sim, }\KeywordTok{c}\NormalTok{(}\StringTok{"V1"}\NormalTok{, }\StringTok{"V10"}\NormalTok{, }\StringTok{"N1"}\NormalTok{, }\StringTok{"N2"}\NormalTok{))}
\KeywordTok{print}\NormalTok{(gp)}
\end{Highlighting}
\end{Shaded}

\includegraphics{PS3_Ko_KueiYueh_v4_files/figure-latex/unnamed-chunk-17-1.pdf}

The results of the impact of noise is showed together with the
simulation in the next question.

\subsubsection{Q1 (d)}\label{q1-d}

\textbf{Perform the two analyses above across a full grid of ICCs and
noise.Evaluate your results. What patterns do you see? Is there an
interaction between noise and ICC?}

simulation with different ICC and noise

\begin{Shaded}
\begin{Highlighting}[]
\NormalTok{### intialization}
\NormalTok{iccs =}\StringTok{ }\KeywordTok{seq}\NormalTok{(}\FloatTok{0.1}\NormalTok{, }\FloatTok{1.0}\NormalTok{, }\DataTypeTok{length.out =} \DecValTok{10}\NormalTok{)}
\NormalTok{qs   =}\StringTok{ }\KeywordTok{c}\NormalTok{(}\DecValTok{0}\NormalTok{, }\DecValTok{1}\NormalTok{, }\DecValTok{25}\NormalTok{, }\DecValTok{49}\NormalTok{, }\DecValTok{50}\NormalTok{)}
\NormalTok{ITER_KM =}\StringTok{ }\DecValTok{500}
\KeywordTok{set.seed}\NormalTok{(}\DecValTok{0}\NormalTok{)}

\NormalTok{### simulation and collect the results}
\NormalTok{res_km_icc_q =}\StringTok{ }\OtherTok{NULL}
\ControlFlowTok{for}\NormalTok{ (icc }\ControlFlowTok{in}\NormalTok{ iccs)\{}
    \KeywordTok{cat}\NormalTok{(}\StringTok{"}\CharTok{\textbackslash{}n}\StringTok{icc:"}\NormalTok{, icc)}
    
    \ControlFlowTok{for}\NormalTok{ (q }\ControlFlowTok{in}\NormalTok{ qs)\{}
        \KeywordTok{cat}\NormalTok{(}\StringTok{"#"}\NormalTok{)}
        
\NormalTok{        dat_sim =}\StringTok{ }\KeywordTok{simulate_data_noise}\NormalTok{(icc, q)}
        
\NormalTok{        unmatch =}\StringTok{ }\KeywordTok{simulate_kmean}\NormalTok{(dat_sim, ITER_KM)}
\NormalTok{        unmatch_sum =}\StringTok{ }\KeywordTok{summary_unmatch}\NormalTok{(unmatch, }\DataTypeTok{icc =}\NormalTok{ icc, }\DataTypeTok{q =}\NormalTok{ q)}
        
\NormalTok{        res_km_icc_q =}\StringTok{ }\KeywordTok{rbind}\NormalTok{(res_km_icc_q, unmatch_sum)}
\NormalTok{    \} }\CommentTok{# end for loop}
\NormalTok{\} }\CommentTok{# end for loop}
\end{Highlighting}
\end{Shaded}

\begin{verbatim}
## 
## icc: 0.1#####
## icc: 0.2#####
## icc: 0.3#####
## icc: 0.4#####
## icc: 0.5#####
## icc: 0.6#####
## icc: 0.7#####
## icc: 0.8#####
## icc: 0.9#####
## icc: 1#####
\end{verbatim}

Visualize the results: when q = 50, the pattern is exactly the same as
we show above, where no parameters has ICC \textasciitilde{} 0. However,
from the number of unmatched events, I supprisingly observed that once a
pure ``noisy'' parameters (parameter with ICC \textasciitilde{} 0) is
added, the kmeans clustering performs very unefficiently. None of the
kmeans clustering results is ``perfect''.
\includegraphics{PS3_Ko_KueiYueh_v4_files/figure-latex/unnamed-chunk-19-1.pdf}

\subsection{Q2 PCA and Clustering}\label{q2-pca-and-clustering}

\textbf{While PCA solves a different problem than clustering it is also
used to find structure in data. }

\begin{Shaded}
\begin{Highlighting}[]
\NormalTok{#### helper function to visualize the PCA results}
\NormalTok{plot_pca_scree <-}\StringTok{ }\ControlFlowTok{function}\NormalTok{(res_pca)\{}
\NormalTok{    ###}
\NormalTok{    x =}\StringTok{ }\NormalTok{res_pca}\OperatorTok{$}\NormalTok{sdev}
\NormalTok{    x =}\StringTok{ }\NormalTok{x}\OperatorTok{^}\DecValTok{2}
\NormalTok{    x =}\StringTok{ }\NormalTok{x }\OperatorTok{/}\StringTok{ }\KeywordTok{sum}\NormalTok{(x)}
  
\NormalTok{    ###}
\NormalTok{    df =}\StringTok{ }\KeywordTok{data.frame}\NormalTok{(}
        \DataTypeTok{PC   =} \DecValTok{1}\OperatorTok{:}\KeywordTok{length}\NormalTok{(x),}
        \DataTypeTok{prop =}\NormalTok{ x, }
        \DataTypeTok{cump =} \KeywordTok{cumsum}\NormalTok{(x))}

\NormalTok{    idx =}\StringTok{ }\KeywordTok{which}\NormalTok{(df}\OperatorTok{$}\NormalTok{cump }\OperatorTok{>}\StringTok{ }\FloatTok{0.9}\NormalTok{)[}\DecValTok{1}\NormalTok{]}

\NormalTok{    ###   }
\NormalTok{    gp1 =}\StringTok{ }\KeywordTok{ggplot}\NormalTok{(df, }\KeywordTok{aes}\NormalTok{(}\DataTypeTok{x =}\NormalTok{ PC, }\DataTypeTok{y =}\NormalTok{ prop))}
\NormalTok{    gp1 =}\StringTok{ }\NormalTok{gp1 }\OperatorTok{+}
\StringTok{       }\KeywordTok{geom_point}\NormalTok{() }\OperatorTok{+}
\StringTok{       }\KeywordTok{geom_linerange}\NormalTok{(}\KeywordTok{aes}\NormalTok{(}\DataTypeTok{ymax =}\NormalTok{ prop, }\DataTypeTok{ymin =} \DecValTok{0}\NormalTok{)) }\OperatorTok{+}
\StringTok{       }\KeywordTok{labs}\NormalTok{(}\DataTypeTok{x =} \StringTok{"Number of PC"}\NormalTok{,}
            \DataTypeTok{y =} \StringTok{"Proportion of}\CharTok{\textbackslash{}n}\StringTok{explained variance"}\NormalTok{)}

\NormalTok{    gp2 =}\StringTok{ }\KeywordTok{ggplot}\NormalTok{(df, }\KeywordTok{aes}\NormalTok{(}\DataTypeTok{x =}\NormalTok{ PC, }\DataTypeTok{y =}\NormalTok{ cump))    }
\NormalTok{    gp2 =}\StringTok{ }\NormalTok{gp2 }\OperatorTok{+}
\StringTok{        }\KeywordTok{geom_point}\NormalTok{() }\OperatorTok{+}
\StringTok{        }\KeywordTok{geom_linerange}\NormalTok{(}\KeywordTok{aes}\NormalTok{(}\DataTypeTok{ymax =}\NormalTok{ cump, }\DataTypeTok{ymin =} \DecValTok{0}\NormalTok{)) }\OperatorTok{+}
\StringTok{        }\KeywordTok{labs}\NormalTok{(}\DataTypeTok{title =} \KeywordTok{paste}\NormalTok{(}\StringTok{"#PC to explained 90% of variance:"}\NormalTok{, idx),}
             \DataTypeTok{x =} \StringTok{"Number of PC"}\NormalTok{,}
             \DataTypeTok{y =} \StringTok{"Cumulative proportion of}\CharTok{\textbackslash{}n}\StringTok{explained variance"}\NormalTok{)}
    
    \KeywordTok{return}\NormalTok{(}\KeywordTok{list}\NormalTok{(gp1, gp2))}
\NormalTok{\} }\CommentTok{# end func}


\NormalTok{#### helper function to visualize the PCA results}
\NormalTok{plot_pca_score <-}\StringTok{ }\ControlFlowTok{function}\NormalTok{(res_pca, }\DataTypeTok{vec_for_color =} \OtherTok{NULL}\NormalTok{)\{}
    
\NormalTok{    x =}\StringTok{ }\NormalTok{res_pca}\OperatorTok{$}\NormalTok{sdev}
\NormalTok{    x =}\StringTok{ }\NormalTok{x}\OperatorTok{^}\DecValTok{2}
\NormalTok{    x =}\StringTok{ }\NormalTok{x }\OperatorTok{/}\StringTok{ }\KeywordTok{sum}\NormalTok{(x)}
    
    \ControlFlowTok{if}\NormalTok{ (}\KeywordTok{is.null}\NormalTok{(vec_for_color))\{}
\NormalTok{        gp =}\StringTok{ }\KeywordTok{data.frame}\NormalTok{(}
            \DataTypeTok{pc1 =}\NormalTok{ res_pca}\OperatorTok{$}\NormalTok{scores[, }\DecValTok{1}\NormalTok{],}
            \DataTypeTok{pc2 =}\NormalTok{ res_pca}\OperatorTok{$}\NormalTok{scores[, }\DecValTok{2}\NormalTok{]) }\OperatorTok
\StringTok{            }\KeywordTok{ggplot}\NormalTok{(., }\KeywordTok{aes}\NormalTok{(}\DataTypeTok{x =}\NormalTok{ pc1, }\DataTypeTok{y =}\NormalTok{ pc2))}
        
\NormalTok{    \} }\ControlFlowTok{else}\NormalTok{ \{}
\NormalTok{        gp =}\StringTok{ }\KeywordTok{data.frame}\NormalTok{(}
            \DataTypeTok{pc1 =}\NormalTok{ res_pca}\OperatorTok{$}\NormalTok{scores[, }\DecValTok{1}\NormalTok{],}
            \DataTypeTok{pc2 =}\NormalTok{ res_pca}\OperatorTok{$}\NormalTok{scores[, }\DecValTok{2}\NormalTok{],}
            \DataTypeTok{x_col =}\NormalTok{ vec_for_color) }\OperatorTok
\StringTok{            }\KeywordTok{ggplot}\NormalTok{(., }\KeywordTok{aes}\NormalTok{(}\DataTypeTok{x =}\NormalTok{ pc1, }\DataTypeTok{y =}\NormalTok{ pc2, }\DataTypeTok{color =}\NormalTok{ x_col))}
\NormalTok{    \} }\CommentTok{# end if-else}
    
\NormalTok{    gp =}\StringTok{ }\NormalTok{gp }\OperatorTok{+}
\StringTok{        }\KeywordTok{geom_point}\NormalTok{() }\OperatorTok{+}
\StringTok{        }\KeywordTok{labs}\NormalTok{(}\DataTypeTok{x =} \KeywordTok{paste}\NormalTok{(}\StringTok{"PC1:"}\NormalTok{, }\KeywordTok{round}\NormalTok{(x[}\DecValTok{1}\NormalTok{], }\DecValTok{3}\NormalTok{) }\OperatorTok{*}\StringTok{ }\DecValTok{100}\NormalTok{, }\StringTok{"%"}\NormalTok{),}
             \DataTypeTok{y =} \KeywordTok{paste}\NormalTok{(}\StringTok{"PC2:"}\NormalTok{, }\KeywordTok{round}\NormalTok{(x[}\DecValTok{2}\NormalTok{], }\DecValTok{3}\NormalTok{) }\OperatorTok{*}\StringTok{ }\DecValTok{100}\NormalTok{, }\StringTok{"%"}\NormalTok{),}
            \DataTypeTok{title =} \KeywordTok{paste}\NormalTok{(}\StringTok{"Total variation explained:"}\NormalTok{, }
                          \KeywordTok{round}\NormalTok{(x[}\DecValTok{1}\NormalTok{] }\OperatorTok{+}\StringTok{ }\NormalTok{x[}\DecValTok{2}\NormalTok{], }\DecValTok{3}\NormalTok{) }\OperatorTok{*}\StringTok{ }\DecValTok{100}\NormalTok{,}
                          \StringTok{"%"}\NormalTok{))}

    \KeywordTok{return}\NormalTok{(gp)}
\NormalTok{\} }\CommentTok{# end func}
\end{Highlighting}
\end{Shaded}

\subsubsection{Q2 (a)}\label{q2-a}

\textbf{Choose a dataset from 1b where you were able to uncover
clusters. Apply PCA. How many meaningful PCs do you find? Plot PC1 vs
PC2.}

Here I used ICC = 0.3, and the result shows that there are about two
meaningful PCs.

\begin{Shaded}
\begin{Highlighting}[]
\NormalTok{### initalization}
\KeywordTok{set.seed}\NormalTok{(}\DecValTok{0}\NormalTok{)}

\NormalTok{### simulation}
\NormalTok{dat_sim =}\StringTok{ }\KeywordTok{simulate_data}\NormalTok{(}\FloatTok{0.3}\NormalTok{)}

\NormalTok{### PCA analysis}
\NormalTok{dat     =}\StringTok{ }\NormalTok{dat_sim }\OperatorTok\StringTok{ }\KeywordTok{select}\NormalTok{(}\OperatorTok{-}\NormalTok{Group)}
\NormalTok{res_pca =}\StringTok{ }\KeywordTok{princomp}\NormalTok{(dat)}

\NormalTok{### visualization}
\NormalTok{gp =}\StringTok{ }\KeywordTok{plot_pca_scree}\NormalTok{(res_pca)}
\KeywordTok{grid.arrange}\NormalTok{(gp[[}\DecValTok{1}\NormalTok{]] }\OperatorTok{+}\StringTok{ }\KeywordTok{ggtitle}\NormalTok{(}\StringTok{"ICC: 0.3"}\NormalTok{), }
\NormalTok{             gp[[}\DecValTok{2}\NormalTok{]])}
\end{Highlighting}
\end{Shaded}

\includegraphics{PS3_Ko_KueiYueh_v4_files/figure-latex/unnamed-chunk-21-1.pdf}

The number PCs required to explain 90\% of total variance decrease when
ICC increase

\begin{Shaded}
\begin{Highlighting}[]
\NormalTok{### initalization}
\KeywordTok{set.seed}\NormalTok{(}\DecValTok{0}\NormalTok{)}

\NormalTok{### simulation}
\NormalTok{dat_sim =}\StringTok{ }\KeywordTok{simulate_data}\NormalTok{(}\FloatTok{0.9}\NormalTok{)}

\NormalTok{### PCA analysis}
\NormalTok{dat     =}\StringTok{ }\NormalTok{dat_sim }\OperatorTok\StringTok{ }\KeywordTok{select}\NormalTok{(}\OperatorTok{-}\NormalTok{Group)}
\NormalTok{res_pca =}\StringTok{ }\KeywordTok{princomp}\NormalTok{(dat)}

\NormalTok{### visualization}
\NormalTok{gp =}\StringTok{ }\KeywordTok{plot_pca_scree}\NormalTok{(res_pca)}
\KeywordTok{grid.arrange}\NormalTok{(gp[[}\DecValTok{1}\NormalTok{]] }\OperatorTok{+}\StringTok{ }\KeywordTok{ggtitle}\NormalTok{(}\StringTok{"ICC: 0.9"}\NormalTok{), }
\NormalTok{             gp[[}\DecValTok{2}\NormalTok{]])}
\end{Highlighting}
\end{Shaded}

\includegraphics{PS3_Ko_KueiYueh_v4_files/figure-latex/unnamed-chunk-22-1.pdf}

\subsubsection{Q2 (b)}\label{q2-b}

\textbf{Repeat but set the number of groups k to be 4. What
relationships do you notice between the number of ``true groups'' and
the number} of meaningful components.


\end{document}
